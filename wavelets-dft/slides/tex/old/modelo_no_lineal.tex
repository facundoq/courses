\begin{myframe}
\pgfmathsetseed{1}
\centering
\resizebox{1\textwidth}{!}{
\begin{tikzpicture}[]
\begin{axis}[xlabel={Amplitud},ylabel={Duración}, yticklabels={,,},xticklabels={,,}]
\datosnolineal
\draw[thick,color=black] (axis cs:-4,-10) -- (axis cs:2,10);
\draw[thick,color=black] (axis cs:-10,-2.5) -- (axis cs:10,2.5);
\end{axis}
\end{tikzpicture}
}
\end{myframe}


\begin{myframe}
\pgfmathsetseed{1}
\centering
\resizebox{1\textwidth}{!}{
\begin{tikzpicture}[]
\begin{axis}[xlabel={Amplitud},ylabel={Duración}, yticklabels={,,},xticklabels={,,}]
\datosnolineal
\draw[thick,color=surface] (axis cs:-4,-10) --
 (axis cs:2,10);
\draw[thick,color=surface] (axis cs:-10,-2.5) -- 
(axis cs:10,2.5);
\node[color=surface] at (axis cs:4,2.5) {\huge{+}};
\node[color=surface] at (axis cs:-3.5,-4.5) {\huge{+}};
\end{axis}
\end{tikzpicture}
}
\end{myframe}



\begin{myframe}
\pgfmathsetseed{1}
\centering
\resizebox{1\textwidth}{!}{
\begin{tikzpicture}[]
\begin{axis}[xlabel={Amplitud},ylabel={Duración}, yticklabels={,,},xticklabels={,,}]
\datosnolineal
\draw[thick,color=black] (axis cs:-4,-10) -- (axis cs:2,10);
\draw[thick,color=black] (axis cs:-10,-2.5) -- (axis cs:10,2.5);
\node[color=surface] at (axis cs:0,0) {\huge{+}};
\addplot[samples=50,domain=-3.6:6,thick,color=surface] 
{(ln(x+4)*3-5)};
\end{axis}
\end{tikzpicture}
}
\end{myframe}


\begin{myframe}
\centering
%\documentclass{standalone}
%\usepackage{tikz}
%\usetikzlibrary{matrix,chains,positioning,decorations.pathreplacing,arrows}

%\begin{document}
    \tikzstyle{every node}=[font=\large]

\begin{tikzpicture}[
plain/.style={
  draw=none,
  fill=none,
  },
net/.style={
  matrix of nodes,
  nodes={
    draw,
    circle,
    inner sep=10pt
    },
  nodes in empty cells,
  column sep=2cm,
  row sep=4pt
  },
>=latex
]
\matrix[net] (mat)
{
$x$ & |[plain]| \\
|[plain]| &  $ f $ \\
$y$ & |[plain]| \\
};

%\draw[<-] (mat-1-1) -- node[above left] {x} +(-2cm,0);
%\draw[<-] (mat-3-1) -- node[above left] {y} +(-2cm,0);


\draw[->] (mat-1-1) --  node[below] {$w_x$} (mat-2-2);
\draw[->] (mat-3-1) -- node[below] {$w_y$} (mat-2-2);
    
\draw[->] (mat-2-2) -- node[above right] {$f(x,y)=S(w_x x + w_y y + b)$} +(2cm,0);

\end{tikzpicture}


%\end{document}

\vspace{-35px}
\begin{columns}

\begin{column}{0.5\textwidth}
\vspace{70px}
\begin{block}{ \centering $S$ es \textbf{no lineal}}
\centering
\pause
Generalmente una sigmoide $\rightarrow$
\end{block}
\end{column}

\begin{column}{0.5\textwidth}
\resizebox{0.75\textwidth}{!}{

\begin{tikzpicture}[]
\begin{axis}[xlabel={x},ylabel={S(x)}]
\addplot[samples=50,domain=-10:10,thick] 
{ 1/(1+e^(-x)) };
\end{axis}
\end{tikzpicture}
}
\end{column}

\end{columns}
\end{myframe}
